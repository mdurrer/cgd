\documentclass[a4paper,12pt]{book}
\renewcommand{\familydefault}{\sfdefault}
\usepackage{helvet} 
\usepackage[utf8]{inputenc}
\usepackage{setspace}
\usepackage[a4paper,left=2.5cm, right=2.0cm]{geometry}
\usepackage{amsfonts}
\usepackage{listings}
\lstset{language=C}
\usepackage[ngerman]{babel}
\usepackage[T1]{fontenc}
\pagestyle{headings}
\usepackage{graphicx}
\usepackage{rotating}
\usepackage{subfigure}
\usepackage{makeidx}
\onehalfspacing
%opening
\title{Die Krankenkasse}
\author{Michael Durrer}
\makeindex
\begin{document}
\begin{titlepage}
\begin{center}
\begin{Large}\hspace{2cm} Vertiefungsarbeit \end{Large}
\begin{Huge}
\newline `Die Krankenkasse am Scheideweg`
\end{Huge}
\includegraphics[40,40][400,400]{medizin.jpg}
\newline
 \begin{large}

\begin{tabular}{|l|r|}\hline
Titel & \emph{Die Krankenkasse am Scheideweg}\\\hline\hline
Autor & \emph{Michael Durrer}\\\hline\hline 
Lehrperson & \emph{Sandra Haller}\\\hline\hline
Schule & \emph{Berufsbildung Baden}\\\hline\hline
Abgabedatum & \emph{25.04.2007}\\\hline

\end{tabular}
 \end{large}
\end{center}

 \end{titlepage}

\pagenumbering{arabic}
\tableofcontents

%\begin{abstract}

%\end{abstract}
\setcounter{page}{0}
\thispagestyle{empty}
\chapter{Vorwort}

Diese Vertiefungsarbeit (im Nachfolgenden \textit{VA} genannt) handelt von der Einheitskrankenkasse, respektive der aktuellen Diskussion um den Wechsel von den 87 verschiedenen Krankenkassen, die sich in einem freien Markt konkurrieren, zu einer Einheitskrankenkasse, die für alle Schweizer Bürger die Grundleistungen abdecken soll. \\
Vorweg möchte ich darauf hinweisen: Die VA selber wurde mir als Übungsarbeit zugetragen von meiner Gesellschaftslehrerin \textit{Frau Sandra Haller}; daher blieb mir keine freie Wahl des Themas. Da ich jedoch selber sehr interessiert bin an diesem Thema, fällt mir die Arbeit daran nicht weiter schwer. Im Verlauf dieser Arbeit werde ich auf das Krankenkassen-System der Schweiz eingehen, bzw. der aktuelle Situation, sowie den weiteren Verlauf bei den jeweiligen Abstimmungsergebnissen.\\
Es sollen, soweit möglich, bisherige und mögliche zukünftige Situationen beleuchtet werden, um einen möglichst flächendeckenden Einblick in das Schweizerische Gesundheits- und  Krankenkassensystem zu gewährleisten.
\\
Ich hoffe, dass meine VA für den Leser interessant ist und zu seiner Entscheidung bei der Abstimmung eine interessante Rückbetrachtung ermöglichen kann, in welche Richtung es auch gewesen sei, ob \textit{ja} oder \textit{nein}.
\\
In meiner Arbeit über die Einheitskrankenkasse schaue ich die Thematik in verschiedenen Perspektiven an, mehrere Aspekte und die Beantwortung dazugehöriger Fragen sowie separater Informationen werden im Auftrag gefordert:
\begin{itemize}
\item Aspekt Geschichte\newline Seit wann gibt es die Krankenkasse?
\item Aspekt Identität\newline
Wie funktioniert die Krankenkasse?\newline
Welche Änderungen würde eine Einheitskrankenkasse bringen?
\item Aspekt Politik\newline Wie funktioniert überhaupt eine Abstimmung?\newline
Auswertung der Abstimmungsresultate\newline
\end{itemize}
Von den Aspekten unabhängig (oder nur teilweise betroffen): \newline Eigene Meinung zum Thema\newline Meinung einer ausgewählten Person zum Thema\newline

\newpage

\chapter{Die Abstimmung über die Einheitskrankenkasse}
In diesem Kapitel gehe ich auf die spezifischen Abschnitte der Abstimmung ein. Es soll darüber informiert werden, worum es überhaupt geht, welche Möglichkeiten sich bieten mit den verschiedenen Abstimmungsmöglichkeiten und es wird berichtet über die Folgen der Abstimmung und das Resultat davon.\\
Zusätzlich gehe ich darauf ein, weshalb es zum Abstimmungsresultat überhaupt kommen konnte und analysiert. Desweiteren gehe ich auch darauf ein, aus Gründen der besseren Übersicht und des Verständnisses, wie eine Abstimmung funktioniert. Dies ist spezifisch im Auftrag gefordert und hat daher den Sekundärgrund, diesen Teilauftrag zu erfüllen.
\section{Welche Folgen hätte eine Einheitskrankenkasse?}
Mit der Einführung der Einheitskrankenkasse, ergäbe sich ein grösserer struktureller Wandel im schweizerischen Gesundheitswesen. Die bisherige Vielzahl an Krankenkassen würde dezimiert auf eine einzige, staatliche Institution, die im Anschluss die bisherigen Aufgaben der Krankenkassen übernimmt. Zwar würden die bisherigen Krankenkassen immer noch Zusatzversicherungen anbieten dürfen, aber die Grundversicherung, das Hauptgeschäft, wird verlagert auf die Einheitskrankenkasse.\\
Angestrebt wird damit ein stärkerer Schutz von Minderheiten und (daher auch tendenziell) ärmeren Einwohnern anhand einer fairen Anpassung der Krankenkassenbeiträgen an Einkommen und Vermögen des Versicherten. So hätten reichere Einwohner höhere Prämien zu bezahlen, während Leute mit geringem Einkommen und Vermögen geringere Beiträge einzahlen müssten und so das Armuts- und Existenzrisiko gesenkt wird.\\
\section{Wie funktioniert überhaupt eine Abstimmung?}
Diese Sektion schaut die Abstimmung unter dem \textbf{politischen Aspekt} an und legt somit grossen Wert auf diese Perspektive. Daher kann es sein, dass andere Ansichten etwas kürzer kommen zugunsten der vorher genannten Perspektive. Daher bitte ich den Leser, dies zu berücksichtigen.\\
\subsection{Die Abstimmung in Hinblick auf die Einheitskrankenkasse}
Eine  Abstimmung besteht aus mehreren Teilschritten.
Die Abstimmung über die Einheitskrankenkasse wurde initiiert über eine eine \emph{Initiative}. Innert der gesetzlichen Rahmenfrist von 18 Monaten wurden die benötigten 100'000 Stimmen gesammelt und an die Bundesverwaltung übergeben.
Danach berät der Bundesrat und Parlament darüber; für die Initiative gibt es 3 mögliche Entscheidungen: abgelehnt, angenommen oder abgelehnt mit Gegenvorschlag. Wird sie von diesen Instanzen angenommen, geht es weiter über die letzte Instanz, die Volksabstimmung.\\

Im Gegensatz zu anderen demokratischen Ländern besitzt die Schweiz eine \emph{direkte Demokratie}, was bedeutet, dass hier nicht bloss Abgeordnete ein Einspracherecht zu politischen Fragen haben, sondern auch das Volk ein Mitspracherecht hat. Wenn eine Mehrheit des Volkes die Initiative dann annimmt, ist die Initiative endgültig durch und es werden daraufhin die in der Initative geforderten Gesetze umgesetzt in die Bundesverfassung.\newline
\section{Abstimmungsergebnis}

\chapter{Das Schweizerische Gesundheitssystem}
Dieses Kapitel enthält grundlegende Informationen über das Gesundheitswesen der Schweiz und die Struktur der Institutionen in diesem Bereich.\\\newline
\section{Einleitung zur Thematik}
Die folgende Beschreibung ist nur ein grober Umriss zum Gesundheitssystem, genauere Detailangaben folgen in den nachfolgenden Sektionen.\newline
Das Schweizer Gesundheitssystem als Ganzes sorgt für gesundheitlich dauerhaft oder temporär benachteiligte Menschen und Menschen im Ruhestand. Darunter fallen beispielsweise körperlich oder geistig behinderte Menschen, die nicht oder nur eingeschränkt arbeitsfähig sind und daher von der \emph{Invalidenversicherung} eine \textbf{Invalidenrente} oder unter Umständen eine \textbf{Teilinvalidenrente} erhalten. Bei den temporär benachteiligten Menschen kommen die obligatorischen Grundversicherungen zum Einsatz und zahlen zum  Beispiel \emph{Erwerbsersatz} (solange man nicht arbeitsfähig ist) oder einen Anteil an die Arzt- und Krankenhauskosten bei Unfällen, die von den Grundleistungen der Krankenkasse abgedeckt werden. Beachtet werden muss, dass auch Mutterschaft unter eine \textit{temporäre Benachteiligung} fällt und von der Krankenkasse die Kosten für den Spitalaufenthalt bei der Entbindung und Voruntersuchungen von der Grundversicherung mitbezahlt werden. Spezielle, oftmals seltene Krankheiten/Unfälle müssen je nachdem über eine \emph{Zusatzversicherung} abgedeckt werden; ansonsten muss der Patient die Arzt- und Spitalkosten komplett selber übernehmen.
\newpage
\section{Über das Schweizerische Gesundheitssystem}
In der Schweiz sorgt das Gesundheitssystem für die Bevölkerung bei gesundheitlichen Problemen bei Behinderungen, im Alter oder überhaupt bei (geringfügigen oder schwereren) Unfällen, die die Gesundheit temporär oder langfristig beeinflussen.
Das Gesundheitswesen an sich besteht aus vielen Einzelkomponenten, darunter sind einige staatlich und andere wiederum privat.
\subsection{Gesetzliche Grundlage}
Als Vorlage für den Aufbau und die obligatorischen Bestandteile des Gesundheitssystems dient das Gesetz. So schreibt das \emph{Krankenversicherungsgesetz}(im Nachfolgenden \textbf{KVG} genannt) gewisse Aspekte vor. So muss beispielsweise jede in der Schweiz wohnende Person (Schweizer wie Ausländer) bei einer \emph{Krankenpflegeversicherung} (auch \emph{Grundversicherung} genannt) versichert sein; es ist \textbf{obligatorisch}, also folglich zwingend.
\section{Historie der Krankenkasse}
Bereits im Vorfeld der ersten `richtigen` Krankenkassen gab es viele Formen von Hilfsgesellschaften oder Krankenkassen-ähnlichen Organisationen (inkl. Versicherungen) und dies bereits seit Jahrhunderten. Tiefergehend befasste man sich erst im 19. Jahrhundert damit. Die Deutschen errichteten \emph{1883/1884} die ersten Versicherungsgesetze in der deutschsprachigen Kultur Europas und begründete damit das moderne Gesundheitswesten. Motiviert durch die \emph{`deutschen Kranken- und Unfallversicherungsgesetze`}, befasste sich nun auch die schweizerische Gesetzgebung (das Parlament, Nationalrat und Ständerat) seit den 80er Jahren des 19. Jahrhunderts mit dieser schwierigen Thematik.\\

\subsection{Die ersten Schweizer Versicherungsgesetze}
5 Jahre nach der Motion\footnote{Eine Motion ist eine bestimmte Art von Parlamentarischem Vorstoss auf eidgenössischer, kantonaler oder kommunaler Ebene. $(http://de.wikipedia.org/wiki/Motion_(Schweiz) [21.3.2007])$} von Nationalrat \emph{Wilhelm Klein} entstand \emph{1890} ein Artikel der Bundesverfassung, welcher den Bund ermächtigte zur Einrichtung einer Kranken- und Unfallversicherung. Entsprechend den deutschen Gesetzen, das Kranken- und Unfallversicherungsgesetz wurde als Vorlage genommen, entwarf man nun in den folgenden Jahren ebenfalls ein eigenes \emph{Kranken- und Unfallversicherungsgesetz} (\emph{KUVG}. Das KUVG von \emph{1899} scheiterte jedoch \emph{1900} am  Referendum politischer Aussenseiter.\\

\subsection{Die Etablierung des KUVG \& KVG}
Letztendlich ergab sich in einem zweiten Versuch, das KUVG durchzubringen, doch noch ein Erfolg. \emph{1911} setzte sich eine leicht abgeänderte Fassung des ursprünglichen KUVG bei der Abstimmung durch. Es bildete bis \emph{1995} die erfolgreiche Grundlage für die schweizerischen Krankenkassen. Das KUVG setzte einige wichtige Rahmenbedingungen vor: Der Beitritt zu einer Krankenkasse erfolgte freiwillig und Kassen, die unter Aufsicht des Bundes standen, erhielten Unterstützungen.\\
Jahrzehnte lang blieben die Versuche einer Revisionierung des KUVG erfolglos; schlussendlich wurde es nicht mehr direkt geändert sondern geradewegs ersetzt (im Jahr \emph{1996}) durch das neue \emph{Krankenversicherungsgesetz} (\emph{KVG}).\\

Als  besonderes Novum enthielt das KVG das bei einer Abstimmung im Jahr \emph{1900} verworfene Obligatorium, welches vorschrieb, dass man eine Grundversicherung besitzen muss.
Zusätzlich enthielt das Gesetz die Geschlechtergleichstellung, u.A. in Anbetracht der zu zahlenden Prämien.
Desweiteren wurden Leute mit kleinem Einkommen seither subventioniert im  Bezug auf die Versicherungsprämien und der Wechsel zu einer anderen Krankenkasse wurde erleichtert.
\chapter{Die Krankenkassen und das Gesundheitssystem}
Dieses Kapitel umfasst zum Einen die existierenden Krankenkassen und die gesetzlichen Rahmen- und Randbedingungen an die sie gebunden sind. Zum Anderen wird die Krankenkasse an sich analysiert und durchleutet, sprich: Die Struktur einer Krankenkasse wird erklärt.
Dieses Kapitel handelt ausserdem vom schweizerischen Gesundheitssystem und geht auf einzelne Aspekte davon ein. Darunter fallen:\newline
\begin{itemize}
\item Aufbau und Struktur
\item Kosten des Gesundheitssystems
\item Rolle der Krankenkassen
\end{itemize}
Diese Aspekte sind ausschlaggebend gewesen für den Ausgang der Abstimmung über die neue Einheitskrankenkasse, daher ist es mir wichtig, dass diese Punkte entsprechend beleuchtet werden.
\section{Allgemeines zu den Krankenkassen}
Die Krankenversicherung in der Schweiz wird von über 100 Versicherern durchgeführt; dies sind die Krankenkassen. Sie alle sind an gesetzliche Rahmenbedingungen gebunden und sind somit in ihrer Handlungsfreiheit marktwirtschaftlich eingeschränkt, es ergibt sich eine sogenannte \emph{Scheinkonkurrenz}.\\
So dürfen Krankenkassen beispielsweise nicht nach Gewinn streben, denn dies könnte zum Nachteil der versicherten Personen ausfallen, da Leistungen plötzlich gestrichen würden (insofern sie nicht gesetzlich vorgeschrieben waren). So müssen die Krankenkassen ausserdem jede in der Schweiz wohnhafte Person ohne Einschränkung(en) in die Grundversicherung aufnehmen, wenn eine Person darum bittet. Die grosse Zahl der Krankenkassen scheint auf den ersten Blick verwirrend, denn die Mindestleistungen, die eine Krankenkasse bietet, sind gesetzlich vorgeschrieben. Jedoch kann sie noch Zusatzversicherungen (z.B. für die Übernahme von Zahnarztkosten) an den Versicherten verkaufen. Falls dies nicht erwünscht ist, kann sie die Kunden noch durch Gratisleistungen zusätzlich zur normalen Grundversicherung anlocken sowie zum Beispiel durch tiefere Prämien als die Mitbewerber, was auch meistens der Hauptgrund für Versicherte ist, ihre Krankenkasse zu wechseln.\\

\section{Geschichte der Krankenkassen}
Krankenkassen gibt es seit langer Zeit; schon im Mittelalter gab es berufs- / zunftbezogene Kassen, die (neben anderen Risiken) das Krankheitsrisiko abdecken sollten. Diese gingen jedoch mehrheitlich dann in den Revolutionsjahren Ende des 18. Jahrhunderts unter.
Dies lies viele Menschen vorerst einmal schutzlos ausgeliefert gegenüber der Industrie und ihren auszehrenden Arbeitsbedingungen und natürlich war der Ruf gross, nach neuen Krankenkassen. Durch diesen Druck gründeten dann auch wieder Anfang des 19. Jahrhunderts Unternehmer und Geistliche neue Kassen; oftmals waren diese Kassen geographisch oder berufsbezogen gebunden, d.h. es wurden zum Beispiel nur Arbeitnehmer eines bestimmten Berufes bei einer Kasse versichert. Besonders beliebt wurden dann die eher allgemein versichernden Fabrik- und Betriebskrankenkassen.
\section{Die Einheitskrankenkasse im Vergleich zum bestehenden System}
Die Einheitskrankenkasss existiert bereits seit einigen Jahren in verschiedenen Ländern, die einen eher sozialistischen / philantropischen Ruf besitzen. Darunter fallen beispielsweise die Niederlande.
In den Niederlanden besitzen Leute mit niedrigem Einkommen die Einheitskrankenkasse, während finanziell Bessergestellte eine Privatkrankenkasse in Anspruch nehmen. Die Einheitskrankenkasse garantiert in diesem Fall, dass arme Leute nicht von der Gesundheitsversorgung ausgeschlossen werden und in eine klassische Abwärtsspirale in eine neue Unterschicht geraten.\\\\
In der Schweiz existiert seit einigen Jahrzehnten bereits das System mit Privatkrankenkassen. Zwar sind die Krankenkassen (im Normalfall zumindest) Privatunternehmen; sie unterstehen jedoch dem \emph{KUVG} (\textit{Kranken- und Unfallversicherungsgesetz}), welches die Kassen verpflichtet, jeden in der Schweiz wohnhafte Person, die in die Kasse aufgenommen werden möchte, zu versichern. Natürlich sind auch die Menschen innerhalb der Schweizer Grenzen dazu verpflichtet, zwingend einer Kasse anzugehören.
Da die Krankenkassen sich gegenseitig konkurrieren und, durch das Gesetz verpflichtet, denselben Leistungskatalog anbieten müssen, unterscheiden die Kassen sich im Wesentlicn nur durch die Höhe ihrer Prämien/Beiträge, die die Versicherten einzahlen müssen.\\
Essentiell sind auch verschiedene Zusatzversicherungen, die von den Kassen angeboten werden. Diese sichern vom Grundleistungskatalog nicht erfasste Krankheiten und Unfälle ab und sind im  Normalfall eher selten benötigt, machen aber in Zeiten rapide ansteigender Behandlungskosten heutzutage durchaus Sinn für die Versicherten.\\
\section{Was ändert sich mit der Einführung der Einheitskrankenkasse?}
Mit der Einheitskrankenkasse ändert sich für den Versicherten im Umgang mit der Krankenkasse nicht allzuviel. Extremer ist dafür die Umstellung der privaten Krankenkassen, die ein positives Wahlergebnis für diese mit sich bringen würde:\\
Die Krankenkassen würden zunächst einmal alle Personen, die bisher ausschliesslich ihre Grundleistungen von ihnen bezogen haben, als Kunden verlieren. Dadurch, dass die Kassen damit automatisch nicht mehr für soviele Leute die Gesundheitskosten zu bezahlen hätten, wäre der bisher höhere Umsatz des Unternehmens natürlich auch nicht mehr zwingend in dieser Grössenmenge nötig. Einen grossen Nachteil hat dies jedoch für den Arbeitsmarkt: Nicht wenige Leute sind heutzutage bei den Krankenkassen angestellt, oftmals Finanz- und Bürokräfte. Durch den geringeren administrativen Aufwand wären auch viele Mitarbeiter nicht mehr notwendig und könnten, bedingt durch den niedrigeren Umsatz, nicht mehr vom Unternehmen gehalten werden.\newline

Der neue Schwerpunkt würde dann für diese Versicherungen im Bereich der \emph{Zusatzversicherungen} liegen, denn viele Krankheitsfälle sind durch den Grundleistungskatalog nicht zwingend abgedeckt, beispielsweise Zahnbehandlungen. Zusatzversicherungen können abgesehen davon auch bei verschiedenen Krankenkassen erworben werden, zu beachten ist hierbei jedoch, dass nicht jede Person  bei einer Zusatzversicherung aufgenommen werden muss und nur die Grundversicherung obligatorisch ist.
\chapter{Schlusswort}
Die ersten Krankenkassen in der Schweiz entstanden im 19. Jahrhundert zur Zeit der Industrialisierung, denn die zunehmende Verarmung und Verelendung der Arbeiter und Handwerker wurde immer mehr zu einem sozialen Problem. Da die ärmeren Schichten durch Krankheiten oder Unfälle in Gefahr liefen, ihre Existenz und die ihrer Nächsten ernsthaft zu gefährden, wurden damals erste Konzepte für eine Krankenkversicherung erarbeitet und umgesetzt.\newline\\
Die Krankenversicherung in der Schweiz fusst auf mehreren konkreten Zielen die angestrebt werden. Sie bietet versicherten Betroffenen von Krankheit, Unfällen und Folgen der Mutterschaft Schutz durch die zugesicherten Leistungen, mit denen ein Teil der Behandlungskosten übernommen werden. Im Gegensatz bezahlt der Versicherte der Krankenkasse, welche die Krankenversicherung anbietet, einen monatlichen Betrag, die sogenannte \emph{Prämie}. Durch die regelmässige Einzahlung von Prämien durch alle Versicherten, werden die Betriebskosten und Ausgaben an die Leistungsbezüger bezahlt; da natürlich nicht alle Versicherten ständig Leistungen beziehen, decken die Prämien die Ausgaben für die Leute, welche diese Leistungen derzeit benötigen.\newline\\
Dadurch, dass die Krankenversicherung in der Schweiz obligatoorisch ist, wird zudem gewährleistet, dass sich alle Einwohner daran beteiligen und das System nicht in sich zusammenfällt aufgrund mangelnder Finanzierung.
Somit ist die Krankenversicherung ein solidarisches System und bietet den Schwächeren Schutz.\newline
Das Schweizerische Gesundheitssystem gehört heutzutage zu den progressivsten Umsetzungen von Sozialsystemen. Das Schweizer Volk bezahlt dafür einen hohen Preis, indem es hohe, seit Jahren steigende Versicherungsbeiträge in Kauf nimmt. Hinzu kommt, dass parallel dazu aufgrund der hohen Betriebskosten und Ausgaben der Krankenkassen zunehmend Leistungen abgebaut werden, die nicht zwingend durch den gesetzlich vorgeschrieben Leistungskatalog übernommen werden müssen von einer Krankenkasse. Hinzu kommt auch noch, dass durch die hohe Zahl an Krankenkassen riesige Unterschiede in Bezug auf Leistungen und Prämien bestehen. Somit hat sich in den letzten Jahren eine Diskrepanz zwischen Leistung und Kosten entwickelt, die für den Bürger nicht mehr nachvollziehbar sind. In der Folge wurde in den letzten Jahren der Ruf nach neuen, alternativen Konzepten laut, welche transparenter und fairer (z.B. nach Einkommen) die Prämien erhebt.\\\newline
Ich selber bin sehr froh, dass wir in der Schweiz eine so umfangreiche Krankenversicherung haben, die auch Minderheiten wie Obdachlose, Mittellose oder Drogenabhängige nicht ausgrenzt, sondern sogar versucht wieder zu integrieren und auf Therapie statt Repression setzt. Nur wenige Länder haben auch nur annähernd ein so modernes und humanes Gesundheitswesen. Zwar bin ich der Ansicht, dass heutzutage fast zuviel Geld des Einkommens in die Gesundheitsvorsorge fliesst, andererseits würde ich auf keinen Fall irgendwelche Leistungen abbauen wollen; das Gegenteil ist der Fall, ich sehe nach wie vor Handlungsbedarf bei der Optimierung der Krankenversicherungen. Heutzutage wird immer noch zuwenig geleistet für psychisch Kranke oder es werden Leute in Psychiatrien gegen ihren Willen weggesperrt und unter Verletzung ihrer Menschenrechte misshandelt und ausgenutzt. \\\newline In der Zukunft stellt sich also, insbesondere als Leistungsgesellschaft, für uns die Frage, wie wir mit den sich nahezu exponentiell ausbreitenden psychischen Krankheiten umgehen wollen und ob wir es uns leisten können, diese weiterhin als vernachlässigbar abzutun und uns nur auf körperliche Leiden zu fokussieren. Gerade in den letzten Jahren haben in Europa Krankheiten wie das Burnout-Syndrom, Depression oder Angsterkrankungen rege Fuss gefasst.\footnote{(http://www.pte.ch/pte.mc?pte=070411017 [17.04.2007])}
Als Antwort auf die Forderungen vornehmlich linker Kreise gaben diese gleich selber eine Antwort: Die Einheitskrankenkasse sollte frischen Wind in das Schweizerische Gesundheitssytsem bringen und die Probleme pragmatisch lösen.
Nachdem die von grossem Medienlärm begleitete Abstimmung im März 2007 dann mit einem eindeutigen Nein mit \emph{71.2 Prozent} der Stimmen abgelehnt wurde, folgte grosse Ernüchterung und die Diskussion ist abgeflaut, aber nicht beendet.\footnote{\begin{small}$http://tinyurl.com/2f7hx3 [15.04.2007]$
\end{small}}
\chapter{Bibliographie}
\section{Texte aus dem Internet}
Unbekannter Autor. Schweizer wollen keine Einheitskrankenkasse.\newline \begin{small}http://tinyurl.com/2f7hx3                       \end{small}(15.03.2007)\newline\\
Unbekannter Autor. Psychische Erkrankungen europaweit auf dem Vormarsch.\newline
\begin{small}
 http://www.pte.ch/pte.mc?pte=070411017
\end{small} (20.03.2007)\newline\\
Unbekannter Autor. Dossier Referendum BG. kant. Spitalkostenbeitrag.\newline
\begin{small}
http://www.juvenet.ch/dossiers/bg\_krankenvers.html
\end{small} (17.04.2007) \newline\\

\section{Interview}

\chapter{Anhang}

\end{document}
